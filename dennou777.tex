%%  地球流体電脳倶楽部 LaTeX2e 用クラス version 7 (dennou777) %%
%%       パッケージマニュアル
%%
%% This is file `dennou777.tex'
%%
%%       Auther: Hitomi-san (a.k.a. Carele Smith)
%%
%%%%%%%%%%%%%%%%%%%%%%%%%%%%%%%%%%%%%%%%%%%%%%%%%%%%%%%%%%%%
\documentclass[report,notitlepage]{dennou777}

\usepackage{shortvrb,bxtexlogo,classico,newpxtext,newpxmath,listings}
\usepackage[hidelinks,unicode=true]{hyperref}
\usepackage[scaled]{beramono}
\MakeShortVerb{\|}
\makeatletter\ifx\dsss@engine l\relax
		\usepackage[jfm_yoko=jlreq,jfm_tate=jlreqv,hiragino-pron,]{luatexja-preset}
	\else
		\usepackage{otf}
\fi\makeatother
\renewcommand{\thefootnote}{\textasteriskcentered\arabic{footnote}}
\lstset{
	basicstyle=\ttfamily,
	numbers=left,
	breaklines=true,
	breakatwhitespace=false,
	frame=leftline,
	backgroundcolor={\color[gray]{.85}},
	numberstyle=\ttfamily\small,
}

\newcommand{\vseven}{Nicole}
\newcommand{\textcode}[1]{{\ttfamily\mdseries #1}}

\Dtitle[地球流体電脳倶楽部 \LaTeXe 用クラス version 7 (Nicole)]
	{地球流体電脳倶楽部 \LaTeXe 用クラス version 7 (Nicole)\\|dennou777.cls|}
\Dauthor{ひとみさん}
\Ddate{\today}

\begin{document}

\maketitle
\thispagestyle{DAheadings}

{\noindent\hfill\bfseries\sffamily Dennou6.sty のマニュアルより\hfill}

{\small Dennou6 \LaTeX スタイルは, \LaTeX を用いた, マニュアル, 論文等の作成に便利な,
マクロ定義 スタイルファイル集である.}

\noindent\hrulefill

|dennou777.cls| version 7.0 (Nicole) は開発版であり、\emph{未完成です}。
それゆえ、このドキュメントも未完成です。開発の協力は常に歓迎しています。
GitHub リポジトリは、\url{https://github.com/Hitomi-San/dennou777} です。

\tableofcontents

\chapter{インストール}

\today 現在の |dennou777.cls| version 7.0 (Nicole) の動作に必要なファイルは、
|dennou777.cls|、|d777helper.sty|、そして |Dennnou6.sty| に付属するファイル
(GitHub リポジトリでは、d6styles の配下にある)一式である。これらのファイル(と
このドキュメント)を「\TeX から見える位置」に配置することで、インストールは完了する。

\chapter{Dennou6.sty との相違点}

前身である |Dennnou6.sty| (以下 Version 6)との相違点を述べる。
これは、|dennou777.cls| version 7.0 (Nicole) (以下\vseven)が開発段階であるため、
\emph{|dennou6.sty| の利用者に積極的に利用してもらい}、不具合を取り除いたり、
機能を追加したりするのを、迅速に行いたいためである。

\vseven は、 Version 6 で提供されていた主なマクロがそのまま
使えるように配慮されている(開発者から見て、使用が推奨できないマクロは除く)。
したがって、以下に示すとおりにプリアンブルを書き換えるだけで、以前の \TeX ソース
をコンパイルできるようになるはずである。

不具合を発見したり、機能改修の要望があれば、是非 GitHub に issue として
投げて頂きたい。

\section{クラスファイルとして提供されるようになった}

\vseven はクラスファイル (.cls) として提供される。これは、Version 6 の
マクロ集が、全てスタイルファイル (.sty) として提供されていたことと対称的である。

クラスファイルとして提供されるので、Version 6 を使っていた文書で
\vseven を読み込むためには、|\documentclass| で |dennou777| を読み込み
(必要に応じてクラスオプションも書き換え\footnote{\ref{classoption}節を参照すること。})、
|\usepackage{Dennou6}| を削除するだけで、基本的に動くはずである
\footnote{従来プリアンブルに書かれていた、「章を変えるときに改ページをしない」ための
コードなどがあると、エラーが発生する。なお、章見出しは成り行きで表示されるように、
クラスファイル内で再定義してある}。

|dennou777.cls| は jlreq \footnote{\url{https://github.com/abenori/jlreq}}
をベースに開発している。したがって、\emph{以前の jreport.cls を読み込んで、
|dennou6.sty| を読み込む方法とは、組版の結果が変わる}。

\begin{figure}[t]
\begin{lstlisting}[caption={最小のソース}]
\documentclass{dennou777}
\begin{document}
Happy, \TeX ing は猫である。
\end{document}
\end{lstlisting}
\end{figure}

\subsection{クラスオプション\label{classoption}}

先述の通り、jlreq をベースに開発しているので、jlreq に渡すことの
できるオプションはすべて受け付ける。すなわち、jlreq のドキュメント
\footnote{ターミナル上で |texdoc jlreq| を実行すると読むことができる。}
を読んでもらいたいのだが、Version 6 から\vseven に移行するにあたって、
特に必要そうなオプションを以下に挙げる。

\begin{description}
\item[paper]
	|paper=a4| で紙面のサイズが A4 版になる。a4 以外にも、a0 から a10、
	b0 から b10 などが指定できる。
\item[jafontsize]
	|jafontsize=12pt| で和文のフォントサイズが 12 pt になる。
\item[report]
	jreport 相当のクラスとなる。
\item[book]
	jbook 相当のクラスとなる。
\end{description}

すなわち、
\[\mbox{\ttfamily \textbackslash documentclass[a4j,12pt]\{jreport\}}\]
という宣言があるファイルは、その宣言を
\[\mbox{\ttfamily \textbackslash documentclass[paper=a4,jfontsize=12pt,report]\{dennou777\}}\]
と書き換えることによって(そして |\usepackage{Dennou6}| を削除すると)、
\vseven を適用させることができる。

\subsection{\LuaLaTeX や\upLaTeX に対応した}

jlreq が\LuaLaTeX や\upLaTeX に対応しているように、
\vseven は\LuaLaTeX や\upLaTeX に対応している。
.tex ソースを一切書き換えることなく、\pLaTeX、\upLaTeX、\LuaLaTeX のいずれでも
処理ができる文書を作ることができる。なお、\pLaTeX や\upLaTeX で処理をする場合、
dvi ウェアにはdvipdfmx を用いる前提となっている。

\section{graphicx と xcolor の自動読み込み}

ソースを書き換えずに3つのエンジンで処理が可能になるようにするため、\vseven は処理を
するエンジンを自動判定して、\pLaTeX か\upLaTeX で処理をしていると判断した場合、
dvipdfmx オプションをつけて graphicx パッケージと xcolor パッケージを読み込む。
したがって、\vseven のオプションには(例え dvi ウェアに dvipdfmx を使用すると
しても)、\emph{|dvipdfmx| オプションをつける必要はない}\footnote{むしろ、|dvipdfmx|
オプションをつけると、その文書を\LuaLaTeX で処理すると(当然だが)正常に処理ができない。}。

\section{削除した機能}

Version 6 が提供していた機能のうち、outdated な機能は削除した。

\subsection{D6graphicx}

今どき PostScript ファイルを直接埋め込む需要がないだろうし、内部で graphicx を
(変なオプションをつけて)読み込んでいるので、すべて削除した。

D6graphicx に相当する機能は、graphicx のマクロを直接利用したり、\TikZ を利用する
などして扱うべきである。



\end{document}