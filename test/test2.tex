\documentclass[book,notitlepage]{dennou777}
%\documentclass[report,]{jlreq}

%%% テスト文書で使うパッケージ

% Option clash チャレンジ
%\usepackage[x11names]{xcolor}

\usepackage[T1]{fontenc}
\usepackage{lmodern}
\usepackage{amsmath,tikz,tcolorbox}
\usetikzlibrary{plotmarks}


%%% ドキュメントクラスを切り替えながらデバクするために
\makeatletter
\ifx t\dsss@dsss@loaded
	\Dtitle{dennou777.cls テスト文書}
	\Dauthor{ひとみ}
	\Ddate{\today}
\fi
\makeatother

%%% 以下テスト文書本体
\begin{document}

\maketitle

この文書は、dennou777.cls のテスト文書である。

\chapter{ベンフォードの法則}

\section{はじめに}
等比数列$\{2^n\}$や、フィボナッチ数列$\{F_n\}$の、「一番上の桁」に
出現しやすい数字は何でしょうか。

例: $2^5=32$なら$3$、$F_7=12$なら$1$。

\section{証明}

\subsection{$\{2^n\}$の場合}

$2^k$の最上位の数字を$d$、$m$を整数とすると、
\begin{gather*}
d\times10^m\leq2^k<(d+1)\times10^m
\intertext{常用対数を取ると、}
m+\log[d]\leq k\log[2]<m+\log[d+1] \\
\frac{m+\log[d]}{\log[2]}\leq k<\frac{m+\log[d+1]}{\log[2]}
\end{gather*}
$d=d_n$のとき、不等式を満たす$k$の幅$x_n$を求める。
\begin{align*}
x_n&=\frac{m+\log[d_n]}{\log[2]}-\frac{m+\log[d_n+1]}{\log[2]} \\
&=\frac{1}{\log[2]}\cdot\log\left[\frac{d_n+1}{d_n}\right]
\end{align*}
これより、$x_n$は$m$(桁数)によらない事がわかる。

後のために$\sum^9_{i=1}[x_i]$を求めておく。
\begin{align*}
\sum^9_{i=1}[x_i]&=\frac{1}{\log[2]}\sum^9_{i=1}\log\left[\frac{d_i+1}{d_i}\right] \\
&=\frac{1}{\log[2]}\cdot\log\left[\frac{2}{1}\cdot\frac{3}{2}\cdot\cdots\cdot\frac{10}{9}\right] \\
&=\frac{1}{\log[2]}\cdot\log[10]=\frac{1}{\log[2]}
\end{align*}
$k$は一様分布なので、$d=n$となる確率$\Pr[d=n]$は、
\begin{align*}
\Pr[d=n]&=\frac{x_n}{\sum^9_{i=1}[x_i]} \\
&=\left.\frac{1}{\log[2]}\cdot\log\left[\frac{n+1}{n}\right]\middle/\frac{1}{\log[2]}\right. \\
&=\log\left[\frac{d+1}{d}\right]
\end{align*}

\subsection{$\{F_n\}$の場合}

\begin{align*}
\begin{cases}
	F_1&=1\\
	F_2&=1\\
	F_{n+2}&=F_{n+1}+F_n
\end{cases}
\end{align*}

フィボナッチ数列の漸化式から一般項を求めると、
\[F_n=\frac{1}{\sqrt{5}}\left(\left(\frac{1+\sqrt{5}}{2}\right)^n-\left(\frac{1-\sqrt{5}}{2}\right)^n\right)\]
である。

ここで、$(1-\sqrt{5})/2 < 1$であるから、$n$が大きいときには
\[F_n\approx \frac{\phi^n}{\sqrt{5}}, \qquad \phi=\frac{1+\sqrt{5}}{2}\,\text{(黄金比)}\]
となり、$2^k$のときと同様の議論によって、同じ確率が導かれる。

\section{導いた式との比較}

$n\leq100$の範囲で、導かれた式と、実際の頻度とを比較する
(表\ref{table}と図\ref{fig})。

\begin{table}[b]
\caption{最上位の数字の頻度\label{table}}
\centering
\begin{tabular}{c|rr}
\hline
数字&$2^n$&$F_n$\\
\hline\hline
$1$&$30$&$30$\\
$2$&$17$&$18$\\
$3$&$13$&$13$\\
$4$&$10$&$9$\\
$5$&$7$&$8$\\
$6$&$7$&$6$\\
$7$&$6$&$5$\\
$8$&$5$&$7$\\
$9$&$5$&$4$\\
\hline
\end{tabular}
\end{table}

\begin{figure}[b]
\centering
\begin{tikzpicture}[x=.08\textwidth,y=.05\textwidth,yscale=20,domain=0:10]
\draw[->](-1,0)--(10,0)node[right]{$x$};\draw(5,-0.05)node[below]{最上位に出現する数字};
\foreach\x in {1,...,9} \draw(\x,0.01)--(\x,-0.01)node[below]{\small$\x$};
\draw[->](0,-0.05)--(0,0.55)node[above]{$y$};\draw(-1,0.25)node[left]{\hbox{\tate 割合}};
\foreach\x in {0.05,0.1,0.15,0.2,0.25,0.3,0.35,0.4,0.45,0.5} \draw(0.2,\x)--(-0.2,\x);
\foreach\x in {0.1,0.2,0.3,0.4} \draw(-0.2,\x)node[left]{\small$\x$};
\draw(0,0) node[below left]{$\mathrm{O}$};
\draw(0.5,0.4) node[above right]{$\displaystyle y=\log_{10}\left[\frac{x+1}{x}\right]$};
\draw [domain=0.4:10,samples=100,smooth] plot(\x,{ln((\x+1)/\x)/ln(10)});
\draw[color=red,thick,mark=text,text mark={\tiny\scalebox{1}[0.05]{●}}] plot file{./2pown.tbl};
\draw[color=red](3,0.3)node[above]{$2^n$};
\draw[color=blue,thick,mark=text,text mark={\tiny\scalebox{1}[0.05]{◆}}] plot file{./fn.tbl};
\draw[color=blue](3,0.3)node[below]{$F_n$};
\end{tikzpicture}
\caption{最上位の数字の頻度\label{fig}}
\end{figure}

\section{Benford's law}
今までの議論は自然界に出現する数値でも、経験則的に成り立つ。

\begin{tcolorbox}[title={Benfordの法則}]
自然界に出現する数値で、最上位に$n$が現れる確率は
\[\log_{10}\left[\frac{n+1}{n}\right]\]
\end{tcolorbox}

\begin{itemize}
\item $n$進法表記の数値に拡張可能。
\item 会計の分野で、不正を見つけるために使われる。
\end{itemize}


\end{document}